\documentclass[12pt, a4paper, oneside, chapter=TITLE, section=title, subsubsection=title, english, brazil, font=plain]{abntex2}

\usepackage{parskip}			% espaçamento entre os parágrafos
\usepackage{microtype} 			% para melhorias de justificação
\usepackage{morefloats}			% permite mais floats

\usepackage[utf8]{inputenc}
\usepackage{amsmath}
\usepackage{indentfirst}
\usepackage{amsfonts}
\usepackage{amssymb}
\usepackage{makeidx}
\usepackage{geometry}
\usepackage{hyperref}
\usepackage{subfig}
\usepackage{tabularx,array,colortbl}
\usepackage{rotating}
\usepackage{multirow}
\usepackage[table]{xcolor}
\usepackage[alf]{abntex2cite}
\usepackage{graphicx}
\geometry{tmargin= 3cm,bmargin= 1.5cm,lmargin= 3cm,rmargin=2.5cm}
\chapterstyle{reparticle}
%\renewcommand{\ABNTEXchapterfont}{\bfseries}
%\renewcommand{\ABNTEXchapterfontsize}{\normalsize}
%\renewcommand{\ABNTEXsectionfontsize}{\normalsize} 
%\renewcommand{\ABNTEXsectionfont}{\bfseries} 
%\renewcommand{\ABNTEXsubsectionfontsize}{\normalsize}
%\renewcommand{\ABNTEXsubsectionfont}{\normalsize}
%\renewcommand{\ABNTEXsubsubsectionfontsize}{\normalsize}
%\renewcommand{\ABNTEXsubsubsectionfont}{\bfseries}
\renewcommand{\labelenumii}{\arabic{enumi}.\arabic{enumii}}

\counterwithout{footnote}{chapter}
\counterwithout{equation}{chapter}
\setlength\afterchapskip{\lineskip}
\hypersetup{bookmarks=true}

\autor{}
\instituicao{ }
\titulo{}
\local{}
\data{}

%\comentario{}
\makeatletter
%\@removefromreset{footnote}{chapter}
\makeatother

\begin{document}
%\selectlanguage{brazil}
%\frenchspacing 

%\imprimircapa
%\imprimirfolhaderosto
%\sumario

% ---
% inserir o sumario
% ---
\pdfbookmark[0]{\contentsname}{toc}
\tableofcontents*
\cleardoublepage
% ---

% ----------------------------------------------------------
% ELEMENTOS TEXTUAIS
% ----------------------------------------------------------
\textual
%\setcounter{page}{2}


\begingroup
\let\clearpage\relax



\chapter{Formato dos dados ambientais de entrada: arquivo data2}

São sete colunas de dados, sendo, da esquerda para direita:\\

\noindent
1\textsuperscript{a} -- datetime: ano, mês, dia e hora YYMMDDHH \\
2\textsuperscript{a} -- Ki(Wm\^{$-2$}): irradiância de onda curta incidente observada [Wm$^{-2}$]\\
3\textsuperscript{a} -- em(hPa): pressão de vapor d'água observada [hPa]\\
4\textsuperscript{a} -- tm(K): temperatura do ar observada [K]\\
5\textsuperscript{a} -- um(ms\^{$-1$}): velocidade horizontal do vento observada [ms$^{-1}$]\\
6\textsuperscript{a} -- prec(mm): precipitação observada [mm]\\
7\textsuperscript{a} -- Rn(Wm\^{$-2$}): saldo de radiação observado [Wm$^{-2}$]\\

O formato explícito em fortran é ``(A8,6F11.4)'', exemplo:

\begin{verbatim}
write(2,'(A8,6F11.4)') nymd, swdown, em, tm, um, prec, rnetm
\end{verbatim}

Os arquivos data2 são gerados pelo programa em Fortran 95 denominado como ``data2\_format.f95'' localizado no diretório ``utils'' do repositório SiB. Inicialmente Rn ficava na 3\textsuperscript{a} coluna, mas em agosto de 2019, a ordem das colunas foi alterada para permitir a inserção de uma condição if no código ``sib2c'' que seleciona o flag ilw e decide sobre o uso da coluna de Rn. 



\chapter{Equação rsoil no Sib2xb}
\vspace{\baselineskip}

O LCB vem utilizando 2 equações para rsoil:

\begin{equation}
  rsoil =  amax1 (0.1, 694. - fac*1500.) + 23.6
\label{default}
\end{equation}

\begin{equation}
  rsoil =  amax1 (0.1, 1001. - exp(fac*6.686))
\label{cinthiacerrado}
\end{equation}


A tabela abaixo indica qual a equação rsoil tem sido utilizada para cada local de medidas.

\begin{table}[ht]
  \centering
  \caption{Equação rsoil utilizada em cada sítio de medidas}
  \begin{tabular}{l c}
    \hline 
    \hline 
  Locais & equação \\
    \hline 
  Cana & \ref{cinthiacerrado}\\
  Cerrado & \ref{cinthiacerrado}\\
  Eucalipto & \ref{default}\\
  Fazenda K77 & \ref{cinthiacerrado}\\ 
  Floresta Atlântica & \ref{default}\\
  Floresta Rondônia & \ref{cinthiacerrado}\\
  Pastagem Rondônia & \ref{cinthiacerrado}\\
  Pastagem SP & \ref{cinthiacerrado}\\
    \hline
  \end{tabular}
\end{table}
\vspace{\baselineskip}

\chapter{Tradução do SiB2 para linguagem C com f2c}
\vspace{\baselineskip}

Pode ser conveniente traduzir o SiB2 do Fortan 77 para linguagem C e para isto pode-se utilizar o f2c (http://www.netlib.org/f2c/). Este processo tem se mostrado eficiente, evitando incompatibilidades de compiladores mais antigos de fortran para os mais atuais. Desta forma obtém-se o código em C e compila-se com o um compilador C, gerando um binário executável independente de um compilador de Fortan.

Inicialmente os arquivos .for devem ser renomeados para .F. Após execute.

\begin{verbatim}
f2c -Nn802 *.F
\end{verbatim}

Desta forma obtém os códigos fontes em C: sib2x.c, Sib2xa.c e Sib2xb.c. Para obter o executável, aqui nomeado com SiB2runF2C, execute.  

\begin{verbatim}
gcc -c -o sib2x.o sib2x.c
gcc -c -o Sib2xa.o Sib2xa.c
gcc -c -o Sib2xb.o Sib2xb.c
gcc -o SiB2runF2C sib2x.o Sib2xa.o Sib2xb.o -lf2c -lm
\end{verbatim}

Este mesmo resultado obtém utilizando o script fort77 em perl que pode ser obtido pelo pacote ``fort77 - Invoke f2c like a real compiler'' do debian.

\vspace{\baselineskip}

\chapter{Usando o GNU Data Language - GDL}
\vspace{\baselineskip}

Baixe a ultima versão em https://github.com/gnudatalanguage/gdl

\begin{verbatim}
mkdir gdl-build
cd gdl-build
cmake -DLIBPROJ4=YES -DLIBPROJ4DIR='/home/user/GDL/libs/proj-4.9.3_bin' -DPYTHON_MODULE=YES ../gdl-mastercmake -DLIBPROJ4=YES -DLIBPROJ4DIR='/home/evandro/pacotes/GDL/libs/proj-4.9.3_bin' -DPYTHON_MODULE=YES -DPYTHON=YES ../gdl-master

\end{verbatim}


\vspace{\baselineskip}


\endgroup

%\renewcommand{\bibname}{REFER\^ENCIAS}
%\bibliography{referencias}

\end{document}